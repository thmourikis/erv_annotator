\documentclass[12pt]{article}
\usepackage[a4paper]{geometry}
\usepackage{url}
\usepackage{alltt}
\usepackage{xspace}
\usepackage{times}
\usepackage{bbm}
\usepackage{verbatim}
\usepackage{ifthen}
\usepackage{optionman}

\newcommand{\Showrep}[2]{\begin{array}{@{}c@{}}#1\\ {#2} \end{array}}
\newcommand{\Function}[3]{#1:#2\to#3}
\newcommand{\Size}[1]{|#1|}
\newcommand{\Iff}{if and only if\xspace}
\newcommand{\Repfind}[0]{\texttt{\small repfind}\xspace}
\newcommand{\Suffixerator}[0]{\texttt{\small suffixerator}\xspace}
\newcommand{\Nats}{\textrm{I}\!\textrm{N}}
\newcommand{\Substring}[3]{#1[#2..#3]}
\newcommand{\Subchar}[2]{#1[#2]}

\title{\texttt{repfind}: a program for finding exact maximal repeats\\
       a manual}

\author{\begin{tabular}{c}
         \textit{Stefan Kurtz}\\
         Center for Bioinformatics,\\
         University of Hamburg
        \end{tabular}}

\begin{document}
\maketitle
This manual describes the options of the program \Repfind. It also gives
some examples on how to use it. We postpone the
definition of the basic notions to the appendix.

\Repfind computes all maximal repeats in
the input sequence represented by the given index,
and echoes them to standard output.
If an error occurs, then the program exits with exit code 1 or 2 (depending on
the kind of error). Otherwise, the exit code is 0. The index is constructed
by the \Suffixerator-program, which is, like \Repfind, part of the
Genometools [1], an open source software for biological sequence analysis.
The \Suffixerator-program can process files in Fasta, Embl, Genbank or
Swissprot format. There can be more than one sequence in any
indexed file.  There are no special options necessary to tell  the
\Suffixerator-program the sequence format. It automatically detects the
appropriate format. We recommend to use either the option \Showoption{dna} or
\Showoption{protein}, depending on the kind of input sequences to be processed.
However, if both of these options are missing \Suffixerator is in most
cases able to automatically identify the kind of sequence. The input for
DNA sequences may, besides the base symbols \(A\), \(C\), \(G\), or \(T\),
contain IUB special characters \(R,Y,M,K,W,S,B,D,H,V,N\). None of the
exact matches reported by \Repfind contains such a character as it
does not match any character not even itself (at a different position).
\Repfind implements a subset of the options of the repfind-program
from the REPuter software suite. While \Repfind (as part of the GenomeTools)
is open source, the REPuter software is closed source.

\section{The Options}

The program \Repfind is called as follows:
\par
\noindent\texttt{gt} \Repfind [\textit{options}] \Showoption{ii}
\Showoptionarg{indexname} [\textit{options}]

The options for \Repfind are as follows:

\begin{Justshowoptions}
\Option{l}{$\ell$}{
Specifies the length parameter $\ell$. This must be a positive integer
smaller than the length of the input sequence. Only repeats of length
at least \(\ell\) are reported. If this option is not used, then
\(\ell=20\).}

\Option{f}{}{Compute maximal forward repeats}

\Option{r}{}{Compute maximal reversed repeats}

\Option{scan}{}{scan the index rather than mapping it to main memory
(as done in the default case). This option only has an effect for computing
forward repeats.}

\Option{ii}{$\Showoptionarg{indexname}$}{Use the index
$\Showoptionarg{indexname}$. This is a mandatory option.}

\Option{v}{}{Be verbose}

\Option{help}{}{display help message and exit}

\end{Justshowoptions}

\subsection*{Important Remark}
The user of the program should carefully choose the length parameter \(\ell\).
The number of maximal repeats exponentially decreases with increasing
\(\ell\).

\section{Output Format}\label{Output}
\Repfind reports maximal repeats \((\ell,i,j\) to the standard output.
Each line of the output in the format

\begin{verbatim}
len seqnum1 relpos1 symbol len seqnum2 relpos2
\end{verbatim}

where
\begin{itemize}
\item
\texttt{S} is either the symbols \texttt{F} or \texttt{R}.
\texttt{F} stands for forward repeats and \texttt{R} stands for
reverse repeats.
\item
\texttt{len} is the length of maximal repeat, which is identical for
the first and the second instance of the repeat. Thus the length is
reported twice.
\item
\texttt{seqnum1} is the sequence number in which the first repeat instance
occurs  at position \texttt{relpos1}.
\texttt{seqnum2} is the sequence number in which the second repeat instance
occurs  at position \texttt{relpos2}. The positions are counted beginning with
0.
\end{itemize}

Suppose the input sequence \(S=gagctcgagcgctgct\)
is contained in the file \texttt{Repfind-example.fna}. Then
\begin{footnotesize}
\begin{verbatim}
gt suffixerator -db repfinf-sample.seq -indexname repidx -dna -suf -tis -lcp -ssp -pl
\end{verbatim}
\end{footnotesize}
creates the index named \texttt{repidx} required for \Repfind. This is
called as follows:

\begin{verbatim}
gt repfind -f -r -l 4 -ii repidx
\end{verbatim}
gives the following output:
\begin{verbatim}
4 0 0 F 4 0 6
4 1 14 R 4 0 12
4 1 11 R 4 0 9
10 1 0 R 10 0 0
5 0 5 R 5 2 7
9 0 0 R 9 2 2
\end{verbatim}
So, for example there is a maximal repeat of length 4  on the forward strand.
Both instances of the repeat are in sequence 0. The first instance begins
are position 0 and the second at position 6.

\section{Remarks}
The programs has been extensively tested, and we are not aware of any bugs.
If the user detects a bug or would like to suggest other options or
features of the program, then please \texttt{kurtz@zbh.uni-hamburg.de}
should be contacted.

\section*{References}
\begin{enumerate}
\item[1]
The GenomeTools genome analysis system \url{http://genometools.org}.
\item[2]
Kurtz, S., Choudhuri, J. V., Ohlebusch, E., Schleiermacher, C., Stoye, J.,
Giegerich, R. (2001). REPuter: The manifold applications of repeat analysis on
a genomic scale. Nucleic Acids Res., 29:4633–4642.
\end{enumerate}

\appendix
\section{Basic Notions}\label{Basic}
We consider a string \(S\) of length \(n\). We index the characters of
\(S\) from \(0\) to \(n-1\), i.e.\ \(S=\Substring{S}{0}{n-1}\). By
\(S^{-1}\) we denote the reverse of \(S\), i.e.\ the string \(u\) of length
\(n\) such that \(\Subchar{u}{i}=\Subchar{S}{n-1-i}\) for
any \(i\in[0,n-1]\).

We define two different kinds of repeats:
\begin{itemize}
\item
\((l,i,j)\) is a \emph{forward or direct repeat} \Iff\ \(i<j\) and
\begin{equation}
\Substring{S}{i}{i+l-1}=\Substring{S}{j}{j+l-1}
\label{forwardEQ}
\end{equation}
\item
\((l,i,j)\) is a \emph{reversed repeat} \Iff \(i\leq j\) and
\begin{equation}
\Substring{S}{i}{i+l-1}=(\Substring{S}{j}{j+l-1})^{-1}
\label{reversedEQ}
\end{equation}
\end{itemize}
\((l,i,j)\) is a \emph{repeat}, if it either is a forward or reversed,
repeat. Each repeat \((l,i,j)\) specifies two \emph{substrings} of \(S\):
\begin{enumerate}
\item
the sequence \(\Substring{S}{i}{i+l-1}\) occurring on the left-hand
side of (\ref{forwardEQ}) and (\ref{reversedEQ}).
This is the \emph{first instance} of \((l,i,j)\).
\item
the sequence occurring on the right-hand
side of (\ref{forwardEQ}) and (\ref{reversedEQ}).
This is the \emph{second instance} of \((l,i,j)\).
\end{enumerate}

By requiring repeats to be \emph{maximal} we can reduce the number of
interesting repeats:
An exact repeat is \emph{maximal} if it is not contained in another
\emph{exact} repeat of the same kind.
\end{document}
